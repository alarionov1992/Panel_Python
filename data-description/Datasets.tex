\documentclass[12pt]{article}

\usepackage[utf8x]{inputenc}
\usepackage[T2A]{fontenc}
\usepackage[english, russian]{babel}

\usepackage{amsmath, amsfonts, amsthm, amssymb, amsopn, amscd}
\usepackage{enumerate}
\usepackage[mathscr]{eucal}

\usepackage{hyperref}
\hypersetup{unicode=true,final=true,colorlinks=true}

\theoremstyle{remark}
%\newtheorem{problem}{Упражнение}
% \newtheorem{problem}{}[section]
% \renewcommand{\theproblem}{\textbf{\#\arabic{problem}}}

% % 
%   Вероятностные определения
%
\DeclareMathOperator{\cov}{cov}
\DeclareMathOperator{\corr}{corr}
\DeclareMathOperator*{\plim}{plim}
\DeclareMathOperator{\Var}{Var}
\DeclareMathOperator{\VVar}{V}

%
%  Суммы квадратов
%
\DeclareMathOperator{\TSS}{TSS}
\DeclareMathOperator{\RSS}{RSS}
\DeclareMathOperator{\ESS}{ESS}

%
%  Эконометрические
%
\DeclareMathOperator{\const}{const}
\DeclareMathOperator{\error}{error}
\DeclareMathOperator{\StdError}{s.e.}
\DeclareMathOperator{\HCStdError}{HC-s.e.}
\DeclareMathOperator{\HACStdError}{HAC-s.e.}
\DeclareMathOperator{\SER}{SER}
\DeclareMathOperator{\DW}{DW}
\DeclareMathOperator{\probit}{probit}
\DeclareMathOperator{\logit}{logit}
\DeclareMathOperator{\gompit}{gompit}
\DeclareMathOperator{\loglog}{loglog}
\DeclareMathOperator{\WhiteNoise}{{WN}}
\DeclareMathOperator{\DurbinWatson}{DW}
\DeclareMathOperator{\VAR}{VAR}
\DeclareMathOperator{\ARMA}{ARMA}
\DeclareMathOperator{\ARIMA}{ARIMA}

%
%  Линейная алгебра
%
\DeclareMathOperator{\rank}{rank}
\DeclareMathOperator{\dimension}{dim}
\newcommand{\LinearSpace}{{\mathfrak L}}


%
%   Числовые
%
\newcommand{\Complex}{{\mathbb C}}
\newcommand{\N}{\mathbb N}
\newcommand{\Z}{{\mathbb Z}}
\newcommand{\Q}{{\mathbb Q}}
\newcommand{\R}{{\mathbb R}}
\newcommand{\semiaxes}{{\mathbb R_+}}

%
%  Вероятностные
%
\newcommand{\iid}{{i.i.d.}}
\newcommand{\Exp}{{\mathsf E}}
\newcommand{\Gauss}{{\mathscr N}}
\newcommand{\Likelihood}{{\mathcal L}}
\newcommand{\StError}{{s.e.}}
\newcommand{\ConfInterval}{{\mathcal I}}

%
%   Вектора
%
\newcommand{\vconst}{{\mathbf const}}
\newcommand{\vectx}{{\mathbf x}}
\newcommand{\vecty}{{\mathbf y}}
\newcommand{\vectz}{{\mathbf z}}
\newcommand{\vecte}{{\mathbf e}}
\newcommand{\vectw}{{\mathbf w}}
\newcommand{\vecth}{{\mathbf h}}
\newcommand{\vectr}{{\mathbf r}}
\newcommand{\vectq}{{\mathbf q}}
\newcommand{\vectf}{{\mathbf f}}%{\boldsymbol{f}}
\newcommand{\vectu}{{\mathbf u}}
\newcommand{\vectv}{{\mathbf v}}
\newcommand{\vectalpha}{{\boldsymbol{\alpha}}}
\newcommand{\vectbeta}{{\boldsymbol{\beta}}}
\newcommand{\vectgamma}{{\boldsymbol{\gamma}}}
\newcommand{\vectdelta}{{\boldsymbol{\delta}}}
\newcommand{\vecteta}{{\boldsymbol{\eta}}}
\newcommand{\vectpi}{{\boldsymbol{\pi}}}
\newcommand{\vectmu}{{\boldsymbol{\mu}}}

% 
%  Матрицы
%
\newcommand{\Id}{I}
\newcommand{\matrixX}{{\mathbf X}}
\newcommand{\matrixY}{{\mathbf Y}}
\newcommand{\matrixU}{{\mathbf U}}
\newcommand{\matrixV}{{\mathbf V}}
\newcommand{\matrixR}{{\mathbf R}}
\newcommand{\matrixZ}{{\mathbf Z}}
\newcommand{\matrixA}{{\mathbf A}}
\newcommand{\matrixB}{{\mathbf B}}
\newcommand{\matrixQ}{{\mathbf Q}}
\newcommand{\matrixH}{{\mathbf H}}
\newcommand{\matrixGamma}{{\boldsymbol{\Gamma}}}
\newcommand{\matrixPi}{{\boldsymbol{\Pi}}}

%
% Теоремы, Примеры etc
%

\theoremstyle{plain}
\newtheorem*{teorema}{Теорема}
\newtheorem*{importante}{Важно!}
\newtheorem*{ejemplo}{Пример}
\newtheorem*{definicion}{Определение}

\theoremstyle{remark}
\newtheorem*{remark}{Замечание}

\title{Описание данных}

\author{Н.В. Артамонов}

\begin{document}

\maketitle

\tableofcontents

% \section{Описание наборов данных}

В этом разделе приведены описания основных наборов данных, используемых для решения задач.

Для работы с нужным датасетом необходимо загрузить его командой
\begin{verbatim}
data(dataset, package)
\end{verbatim}

\section{Кросс-секционные данные}

\begin{table}
	\caption{Набор данных \texttt{Labour}
	из пакета \texttt{Ecdat}  (569  наблюдений) с данными 
	о бельгийских 	фирмах за 1996 г.} 
%	Источник данных \cite{SwissLabor}}
	\label{BelgiumLabour}
	\begin{tabular}{l|l}\hline
	capital & капитал (в млн евро) \\
	labour & число сотрудников  \\
	output &выпуск (в млн евро) \\
	wage  & зарплата на одного сотрудника (в тыс евро)  \\
	\hline
	\end{tabular}
\end{table}

\begin{table}
	\caption{Набор данных \texttt{Electricity}
	из пакета \texttt{Ecdat}  (158  наблюдений) с данными 
	о производителях электроэнергии в US за 1970 г.} 
%	Источник данных \cite{SwissLabor}}
	\label{ElectricityEcdat}
	\begin{tabular}{l|l}\hline
	cost &  общие издержки за год \\
	q & общий выпуск электроэнергии \\
	pl & уровень зарплата (wage rate) \\
	pk & цена привлечения капитала (capital price index) \\
	pf & цена на топливо (fuel price) \\
	\hline
	\end{tabular}
\end{table}


%\paragraph{sleep75}\label{sleep75}

%Набор данных \verb@sleep75@ из пакета \verb@wooldridge@ содержит 706 наблюдений
%по следующим переменным
%\begin{center}
%	\begin{tabular}{l|l}\hline
%	age & возраст (в годах)  \\
%	educ & уровень образования (в года) \\
%	inlf & бинарная, 1 если участник рынка труда \\
%	smsa  & бинарная, 1 если живёт в мегаполисе \\
%	male & гендерный фактор (бинарная, 1 если мужчина) \\
%	marr & семейный статус (бинарная, 1 если женат/заужем) \\
%	prot & бинарная, 1 если протестант \\
%	selfe & бинарная, 1 если самозанятый \\
%	sleep & продолжительность сна (мин/нед) \\
%	south & георафический фактор (бинарная, 1 если живёт на юге) \\
%	spsepay & доход супруга/супруги \\
%	spwrk75 & бинарная, 1 если супруг(а) работает \\
%	totwrk & занятость (мин/нед) \\
%	union & бинарная, 1 если член профсоюза \\
%	yngkid & бинарная, 1 если сеть дети младше 3 лет \\
%	yrsmarr & сколько лет женат/замужем \\
%	hrwage & почасовая оплата \\
%	\hline
%	\end{tabular}
%\end{center}
\begin{table}
	\caption{Набор данных \texttt{sleep75} из пакета \texttt{wooldridge} (706  наблюдений). 
	Основные переменные. Источник данных \cite{sleep75}}
	\label{sleep75}
	\begin{tabular}{l|l}\hline
	age & возраст (в годах)  \\
	educ & уровень образования (в года) \\
	inlf & бинарная, 1 если участник рынка труда \\
	leis1 & нерабочее время, sleep-totwrk \\
	smsa  & бинарная, 1 если живёт в мегаполисе \\
	male & гендерный фактор (бинарная, 1 если мужчина) \\
	marr & семейный статус (бинарная, 1 если женат/замужем) \\
	prot & бинарная, 1 если протестант \\
	selfe & бинарная, 1 если самозанятый \\
	sleep & продолжительность сна (мин/нед) \\
	south & географический фактор (бинарная, 1 если живёт на юге) \\
	spsepay & доход супруга/супруги \\
	spwrk75 & бинарная, 1 если супруг(а) работает \\
	totwrk & занятость (мин/нед) \\
	union & бинарная, 1 если член профсоюза \\
	yngkid & бинарная, 1 если сеть дети младше 3 лет \\
	yrsmarr & сколько лет женат/замужем \\
	hrwage & почасовая оплата \\
	\hline
	\end{tabular}
\end{table}

%Данные взят из статьи \cite{sleep75}



%\paragraph{wage2}
%
%Набор даных \verb@wage2@ из пакета \verb@wooldridge@ содежит 935
%наблюдений по следующим переменным
\begin{table}
	\caption{Набор данных \texttt{wage2} из пакета \texttt{wooldridge} (935  наблюдений).
	Основные переменные. Источник данных \cite{wage2}}
	\label{wage2}
	\begin{tabular}{l|l}\hline
	wage & месячная зарплата \\
	hours & недельная занятость в часах \\
	IQ & результаты теста IQ \\
	KWW & результаты теста knowledge of world work \\
	educ & уровень образования (в годах) \\
	exper & опыт работы в годах \\
	tenure & стаж работы на текущем месте \\
	age & возраст (в годах)  \\
	married & семейный статус (бинарная, 1 если женат/замужем) \\
	south & географический фактор (бинарная, 1 если живёт на юге) \\
	urban & место жительства (1 если живет в городе) \\
	sibs & число братьев/сестёр \\
	brthord & какой по счёту ребёнок в семье \\
	meduc & уровень образования матери (в годах) \\
	feduc & уровень образования отца (в годах) \\
	\hline
	\end{tabular}
\end{table}
%Данные взят из статьи \cite{wage2}

\begin{table}
	\caption{Набор данных \texttt{wage1} из пакета \texttt{wooldridge} (526  наблюдений).
	Основные переменные.} %Источник данных \cite{wage1}}
	\label{wage1}
	\begin{tabular}{l|l}\hline
	wage & средняя почасовая оплата \\
	educ & уровень образования (в годах) \\
	female & гендерный фактор \\
	exper & опят работы \\
	tenure & стаж на текущем месте работы \\
	married & семейный статус \\
	smsa & живёт ли в мегаполисе (бинарная) \\
	south & географический фактор (бинарная) \\
	west & географический фактор (бинарная) \\
	northcen & географический фактор (бинарная) \\
	\end{tabular}
\end{table}

\begin{table}
	\caption{Набор данных \texttt{loanapp} из пакета \texttt{wooldridge} (1989  наблюдений). 
	Основные переменные. Источник данных \cite{loanapp}}
	\label{loanapp}
%\begin{center}
	\begin{tabular}{l|l}\hline
	approve & бинарная, 1 если кредитная заявка одобрена \\
	appinc & доход заявителя  (в \$1000) \\
	mortno & бинарная, 1 если нет ипотечной кредитной истории \\
	unem  & уровень безработицы в отрасли в \% \\
	dep & количество иждивенцев \\
	male & гендерный фактор \\
	married & семейный статус \\
	yjob & стаж на текущей работе \\
	self & бинарная, 1 если самозанятый \\
	\hline
	\end{tabular}
%\end{center}
\end{table}

\begin{table}
	\caption{Набор данных \texttt{SwissLabor} о рынке труда Швейцарии
	из пакета \texttt{AER} (872  наблюдений). 
	Источник данных \cite{SwissLabor}}
	\label{SwissLabor}
%\begin{center}
	\begin{tabular}{l|l}\hline
	participation & Является ли участником рынка труда? \\
	& (фактор, \texttt{''yes''/''no''}) \\
	income & логарифм дополнительного дохода (nonlabor income) \\
	age & возраст (в десятилетиях) \\
	education  & уровень образования \\
	youngkids & число маленьких детей (младше 7 лет)\\
	oldkids & число старшийх детей (страше 7 лет) \\
	foreign & является ли иностранцем?  (фактор, \texttt{''yes''/''no''}) \\
	\hline
	\end{tabular}
%\end{center}
\end{table}

\begin{table}
	\caption{Набор данных из файла \texttt{Mroz}
	из пакета \texttt{Ecdat} содержит 
	данные о рынке труда замужних женщин.
	Основные переменные. Источник данных \cite{Mroz}} 
%	Источник данных \cite{SwissLabor}}
	\label{MrozGreen}
%\begin{center}
	\begin{tabular}{l|l}\hline
	LFP & бинарная, 1 женщина работает \\
	WHRS & уровень занятости (в часах) \\
	KL6 & число детей моложе 6 лет в семье \\
	K618  & число детей от 6 до 18 лет в семье\\
	WA & возраст \\
	WE & уровень образования (в годах) \\
	WW &  средняя почасовая оплата\\
	HHRS & занятость мужа \\
	HA & возраст мужа \\
	HE & уровень образования мужа (а годах) \\
	HW & зарплата мужа \\
	FAMINC & доход домашнего хозяйства \\
	WMED & уровень образования матери \\
	WFED & уровень образования отца \\
	UN & уровень безработицы в стране проживания \\
	CIT & бинарная, 1 если живет в мегаполисе \\
	AX & предыдущий опыт работы (в годах) \\
	\hline
	\end{tabular}
%\end{center}
\end{table}

\begin{table}
	\caption{Набор данных \texttt{diamonds}
	из пакета \texttt{ggplot2} с данными о  бриллиантах
	(53940  наблюдений). Основные переменные.} 
%	Источник данных \cite{SwissLabor}}
	\label{diamonds}
%\begin{center}
	\begin{tabular}{l|l}\hline
	price & цена бриллианта \\
	carat & вес бриллианта (в каратах) \\
	cut & качество огранки (упорядоченный фактор с уровнями \\
	& Fair<Good< Very Good<Premium<Ideal) \\
	color  &  цвет  (упорядоченный фактор с уровнями \\
	& J<I<H<G<F<E<D\\
	clarity & прозрачность (упорядоченный фактор с уровнями \\
	& I1<SI2<SI1<VS2<VS1<VVS2<VVS1<IF\\
	x & длина (в мм) \\
	y &  ширина (в мм) \\
	z & глубина (в мм) \\
	\hline
	\end{tabular}
%\end{center}
\end{table}

\begin{table}
	\caption{Набор данных \texttt{Diamond}
	из пакета \texttt{Ecdat} с данными о  бриллиантах
	(308  наблюдений). Основные переменные.
	Источник данных \cite{Diamonds}}
	\label{Diamond}
%\begin{center}
	\begin{tabular}{l|l}\hline
	carat & вес бриллианта (в каратах) \\
	colour  &  цвет (фактор с уровнями D,E,F,G,H,I) \\
	clarity & прозрачность (фактор с уровнями \\
	& IF,VVS1,VVS2,VS1,VS2) \\
	certification & орган по сертификации (фактор с уровнями \\
	& GIA,IGI,HRD) \\
	price & цена в Сингапуре \\
	\hline
	\end{tabular}
%\end{center}
\end{table}

\begin{table}
	\caption{Набор данных \texttt{nlsw88}
	из пакета \texttt{Counterfactual} о занятости женщин в США
	(2246  наблюдений). Основные переменные.
	Источник данных: сайт Stata \url{http://www.stata-press.com/data/r10/g.html} }
	\label{nlsw88}
%\begin{center}
	\begin{tabular}{l|l}\hline
	hours & недельная занятость (в часах) \\
	married & семейный статус \\
%	never\_married & бинарная \\
	ttl\_exp & общий стаж работы \\
	smsa  & бинарная, 1 если живёт в мегаполисе \\
	south & географический фактор (бинарная, 1 если живёт на юге) \\
	wage & почасовая оплата  (в \$) \\
	age & возраст (в годах)  \\
	grade & уровень образования (в годах) \\
	\hline
	\end{tabular}
%\end{center}
\end{table}


\begin{table}
	\caption{Набор данных из файла \texttt{default.csv}
	с сайта \url{http://meit.mgimo.ru/node/237} о 
	банкротствах по студенческим займам (6778 наблюдений)} 
%	Источник данных \cite{SwissLabor}}
	\label{defaultDataset}
%\begin{center}
	\begin{tabular}{l|l}\hline
	default & бинарная переменная, равная 1 если индивид признал \\
	& себя банкротом по студенческому займу \\
	age & возраст \\
	adepcnt & количество иждивенцев у индивида плюс 1 \\
	acadmos & количество месяцев, которые индивид прожил \\ 
	& по текущему адресу \\
	majordrg & количество зарегистрированных серьёзных \\
	& правонарушений у этого индивида \\
	minordrg &   количество зарегистрированных мелких \\ 
	& правонарушений у этого индивида \\
	ownrent &1, если индивид живёт в собственном дома, и  \\
	& 0, если снимает\\
	income & месячный доход в \$ \\
	spending & среднемесячный расход по кредитной карте \\
	inc\_rep & income, делённая на количество иждивенцев \\
	exp\_inc & доля месячных расходов по кредитной карте \\
	&  в годовой заработной плате \\
	selfempl & 1, если индивид самозанятый, и 0 иначе \\
	\hline
	\end{tabular}
%\end{center}
\end{table}

\begin{table}
	\caption{Набор данных \texttt{stockton3}
	из пакета \texttt{PoEdata}
	 с данными о стоимости домов (2610  наблюдений).  Основные переменные.
	Источник данных \url{https://github.com/ccolonescu/PoEdata} }
	\label{stockton3}
%\begin{center}
	\begin{tabular}{l|l}\hline
	sprice & цена продажи дома  (в \$) \\
	livarea & жилая площадь  (кв.футы) \\
	pool & наличие бассейна (бинарная) \\
	lgelot & размер участка  (бинарный фактор, 1 если участок \\
	& больше 5 акров) \\
	age & возраст (в годах)  \\
	beds & число спален \\
	\hline
	\end{tabular}
%\end{center}
\end{table}

\begin{table}
	\caption{Набор данных из файла \texttt{applications.csv}
	о поступивших на магистерские и PhD-программы (400 наблюдений)} 
%	Источник данных \cite{SwissLabor}}
	\label{applicationsDataset}
%\begin{center}
	\begin{tabular}{l|l}\hline
	admit & бинарный фактор, 1 если заявка одобрена \\
	GPA &средняя оценка за время обучения \\
	GRE & балл за экзамен graduate record exam \\
	rank & категориальная переменная, обозначающая престиж \\
	& университета (1 -- высший престиж, 4 -- низший престиж) \\
	\hline
	\end{tabular}
%\end{center}
\end{table}

\newpage

\section{Временные ряды}

\begin{table}
	\caption{Набор данных \texttt{Consumption}
	из пакета \texttt{Ecdat} об индивидуальных доходах и расходах  в Канаде
	(квартальные  данные  с 1947Q1 по 1996Q4). }
%	Источник данных \cite{Consumption}}
	\label{Consumption}
%\begin{center}
	\begin{tabular}{l|l}\hline
	yd & индивидуальный располагаемый доход в ценах 1986 \\
	ce & индивидуальные расходы на потребление в ценах 1986 \\
	\hline
	\end{tabular}
%\end{center}
\end{table}

\begin{table}
	\caption{Набор данных \texttt{Icecream}
	из пакета \texttt{Ecdat} о потреблении мороженого в США
	(недельные данные  с 1951–03–18 по 1953–07–11, всего 30  наблюдений). 
	Источник данных \cite{Icecream}}
	\label{Icecream}
%\begin{center}
	\begin{tabular}{l|l}\hline
	cons & потребление мороженого (в пинтах) \\
	income  &  средний недельный доход семьи (в \$) \\
	price & цена мороженого (за пинту) \\
	temp & средняя температура (по Фаренгейту) \\
	\hline
	\end{tabular}
%\end{center}
\end{table}

\newpage

\section{Панельные данные}

\begin{table}
	\caption{Панель \texttt{Guns}
	из пакета \texttt{AER} с данными по 51 штату США с 1977 по 1999 гг.
	(всего 1173 наблюдения).  Основные переменные.
	Источник данных \cite{Guns}}
	\label{Guns}
%\begin{center}
	\begin{tabular}{l|l}\hline
	state &  factor indicating state \\
	year  &  factor indicating year \\
	violent & violent crime rate (incidents per 100,000 members of \\
	& the population) \\
	murder & murder rate (incidents per 100,000). \\
	robbery&  robbery rate (incidents per 100,000) \\
	prisoners & incarceration rate in the state in the previous year \\
	&  (sentenced prisoners per 100,000 residents;\\
	& value for the previous year) \\
	afam & percent of state population that is African-American, \\
	& ages 10 to 64 \\
	cauc  & percent of state population that is Caucasian, ages 10 to 64 \\
	male & percent of state population that is male, ages 10 to 29 \\
	population & state population, in millions of people. \\
	income & real per capita personal income in the state (US dollars) \\
	density & population per square mile of land area, divided by 1,000 \\
	law & factor. Does the state have a shall carry law in \\
	& effect in that year?\\
	\hline
	\end{tabular}
%\end{center}
\end{table}

\begin{table}
	\caption{Панель \texttt{LaborSupply}
	из пакета \texttt{plm, Ecdat} с данными по 532 индивидуумам  с 1979 по 1988 гг.
	(всего 5320 наблюдений).  Основные переменные.
	Источник данных \cite{LaborSupply}}
	\label{LaborSupply}
%\begin{center}
	\begin{tabular}{l|l}\hline
	lnhr & логарифм годовой занятости в часах \\
	lnwg & логарифм почасовой оплаты \\
	kids &число детей \\
	age & возраст \\
	disab & бинарная, 1 если плохое здоровье \\
	\hline
	\end{tabular}
%\end{center}
\end{table}


\begin{table}
	\caption{Панель \texttt{Cigar}
	из пакета \texttt{plm} с данными по 46 штатам США с 1963 по 1992 гг.
	(всего 1380 наблюдений).  Основные переменные.
	Источник данных \cite{Cigar}}
	\label{Cigar}
%\begin{center}
	\begin{tabular}{l|l}\hline
	state & state abbreviation \\
	year & the year \\
	price & price per pack of cigarettes \\
	pop & population \\
	pop16 & population above the age of 16 \\
	cpi & consumer price index (1983=100) \\
	ndi & per capita disposable income \\
	sales & cigarette sales in packs per capita \\
	pimin & minimum price in adjoining states per pack of cigarettes \\
	\hline
	\end{tabular}
%\end{center}
\end{table}

\begin{table}
	\caption{Панель \texttt{Gasoline}
	из пакета \texttt{plm, Ecdat} с данными о потреблении бензина по
	18 странам OECD с 1960 по 1978 гг.
	(всего 342 наблюдений).  Основные переменные.
	Источник данных \cite{Gasoline}}
	\label{Gasoline}
%\begin{center}
	\begin{tabular}{l|l}\hline
	lgaspcar & логарифм потребления бензина \\
	lincomep & логарифм реального дохода на душу населения \\
	lrpmg & логарифм реальной цены на бензин \\
	lcarpcap & логарифм объёма рынка машин\\
	\hline
	\end{tabular}
%\end{center}
\end{table}

\begin{table}
	\caption{Панель \texttt{Loan}
	Dream Housing Finance company deals in all home loans. They have a presence across all urban, semi-urban, and rural areas. Customer-first applies for a home loan after that company validates the customer eligibility for a loan.(всего 342 наблюдений).  Основные переменные.
	Источник данных \href{https://www.kaggle.com/datasets/vikasukani/loan-eligible-dataset?select=loan-train.csv}{Loan}}
	\label{Gasoline}
%\begin{center}
	\begin{tabular}{l|l}\hline
	LoanID & Unique Loan ID \\
	Gender & Male/ Female  \\
	Married & applicant married (Y/N) \\
	Dependents & Number of dependents \\
Education & Applicant Education (Graduate/ Under Graduate)\\
Self Employed & Self-employed (Y/N)\\
ApplicantIncome & ApplicantIncome\\
CoapplicantIncome & CoapplicantIncome \\
LoanAmount & LoanAmount \\
Loan Amount Term & Loan Amount Term \\
Credit History & credit history meets guidelines \\
Property Area & Urban/ Semi-Urban/ Rural \\
Loan Status &	Loan approved \\
\end{tabular}
\end{table}
\begin{table}
	\caption{ \texttt{1Custumers.csv}
	Shop Customer Data is a detailed analysis of a imaginative shop's ideal customers. It helps a business to better understand its customers. The owner of a shop gets information about Customers through membership cards. Dataset consists of 2000 records and 8 columns.}
	\label{Gasoline}
%\begin{center}
	\begin{tabular}{l|l}\hline
	Customer ID & --- \\
	Gender & ---\\ 
	Age & --- \\
	Annual Income & --- \\
	Spending Score & Score assigned by the shop,\\
	 & based on customer behaviour and spending nature\\
	Profession & --- \\
	Work Experience & in years\\
	Family Size & --- \\
	\hline
	\end{tabular}
%\end{center}
\end{table}
\begin{table}
	\caption{ \texttt{2laptops.csv}
This is a dataset about laptops scrap from Flipkart which contains information on various aspects of laptops such as their price, discount, specifications, and warranty. The dataset contains a total of 920 entries, each representing a single laptop. It is important to note that there are some missing values in the dataset for some of the columns, including "discount", "In build sw", and "warranty". This should be taken into consideration when analyzing the data. Additionally, all the columns have a data type of "object", which may require further processing to convert the data into a usable format.}
	\label{Gasoline}
%\begin{center}
	\begin{tabular}{l|l}\hline
	title & provides a brief description of the laptop \\
	price &   includes the cost of the laptop \\ 
	discount &  mentions any applicable discounts on the laptop's price\\
	Processor&  specifies the type of processor used in the laptop \\
	RAM &  mentions the amount of RAM the laptop has \\
	OS &  lists the operating system installed on the laptop \\
	SSD &  indicates the size of the solid-state drive\\
	Display &  mentions the screen size and display specifications of the laptop\\
	In build sw &  lists any software pre-installed on the laptop \\
	warranty &  provides information on the warranty offered for the laptop\\
	\hline
	\end{tabular}
%\end{center}
\end{table}

\begin{table}
	\caption{ \texttt{3insurance.csv}
Machine Learning with R by Brett Lantz is a book that provides an introduction to machine learning using R. As far as I can tell, Packt Publishing does not make its datasets available online unless you buy the book and create a user account which can be a problem if you are checking the book out from the library or borrowing the book from a friend. All of these datasets are in the public domain but simply needed some cleaning up and recoding to match the format in the book.}
	\label{Gasoline}
%\begin{center}
	\begin{tabular}{l|l}\hline
age & age of primary beneficiary  \\
sex & insurance contractor gender, female, male  \\
bmi & Body mass index, providing an understanding of body, \\
     & weights that are relatively  high or low relative to height  \\ & objective index of body weight $(kg / m ^ 2)$ \\ &  using   the ratio of height to weight, ideally 18.5 to 24.9 \\
children & Number of children covered by health insurance\\
&  Number of dependents \\
smoker & Smoking \\
region & the beneficiary's residential area in the US, \\
&  northeast, southeast southwest, northwest.\\
charges & Individual medical costs billed by health insurance \\
	\hline
	\end{tabular}
%\end{center}
\end{table}

\begin{table}
	\caption{ \texttt{4winequality-red.csv}
The two datasets are related to red and white variants of the Portuguese "Vinho Verde" wine. For more details, consult the reference [Cortez et al., 2009]. Due to privacy and logistic issues, only physicochemical (inputs) and sensory (the output) variables are available (e.g. there is no data about grape types, wine brand, wine selling price, etc.). These datasets can be viewed as classification or regression tasks. The classes are ordered and not balanced (e.g. there are much more normal wines than excellent or poor ones). This dataset is also available from the UCI machine learning repository,\href{ https://archive.ics.uci.edu/ml/datasets/wine+quality}{winequality} Content
}
%\begin{center}
	\begin{tabular}{l|l}\hline
&	 fixed acidity \\
&	 volatile acidity \\ 
&	 citric acid \\
&	 residual sugar \\
&	 chlorides \\
&	 free sulfur dioxide \\
&	 total sulfur dioxide \\
&	 density \\
&	 pH \\
&	 sulphates \\
&	 alcohol \\
&	Output variable (based on sensory data) \\
&	 quality (score between 0 and 10) \\
	\hline
	\end{tabular}
%\end{center}
\end{table}
\begin{table}
	\caption{ \texttt{5marketing-campaign.csv}
	Customer Personality Analysis is a detailed analysis of a company’s ideal customers. It helps a business to better understand its customers and makes it easier for them to modify products according to the specific needs, behaviors and concerns of different types of customers.
Customer personality analysis helps a business to modify its product based on its target customers from different types of customer segments. For example, instead of spending money to market a new product to every customer in the company’s database, a company can analyze which customer segment is most likely to buy the product and then market the product only on that particular segment. Content: } 
%\begin{center}
People\\
	\begin{tabular}{l|l}\hline

ID & Customer's unique identifier \\
YearBirth & Customer's birth year \\
Education & Customer's education level \\
Marital Status & Customer's marital status \\
Income & Customer's yearly household income \\
Kidhome & Number of children in customer's household \\
Teenhome & Number of teenagers in customer's household \\
Dt Customer & Date of customer's enrollment with the company\\
Recency & Number of days since customer's last purchase \\
Complain & 1 if the customer complained in the last 2 years, 0 otherwise
	\end{tabular}
Products \\
	\begin{tabular}{l|l}\hline

MntWines & Amount spent on wine in last 2 years \\
MntFruits & Amount spent on fruits in last 2 years \\
MntMeatProducts & Amount spent on meat in last 2 years \\
MntFishProducts & Amount spent on fish in last 2 years \\
MntSweetProducts & Amount spent on sweets in last 2 years \\
MntGoldProds & Amount spent on gold in last 2 years 
	\end{tabular}
	
Promotion\\
	\begin{tabular}{l|l}\hline
NumDealsPurchases & Number of purchases made with a discount \\
AcceptedCmp1 & 1 if customer accepted the offer in the 1st campaign, 0 otherwise \\
AcceptedCmp2 & 1 if customer accepted the offer in the 2nd campaign, 0 otherwise \\
AcceptedCmp3 & 1 if customer accepted the offer in the 3rd campaign, 0 otherwise \\
AcceptedCmp4 & 1 if customer accepted the offer in the 4th campaign, 0 otherwise \\
AcceptedCmp5 & 1 if customer accepted the offer in the 5th campaign, 0 otherwise \\
Response & 1 if customer accepted the offer in the last campaign, 0 otherwise
		\end{tabular}
Place\\
		\begin{tabular}{l|l}\hline
NumWebPurchases &Number of purchases made through the company’s website \\
NumCatalogPurchases & Number of purchases made using a catalogue \\
NumStorePurchases & Number of purchases made directly in stores \\
NumWebVisitsMonth & Number of visits to company’s website in the last month\\
	\end{tabular}
%\end{center}
\end{table}


\begin{table}
	\caption{Панель \texttt{6PlacementDataFullClass.csv}
	This data set consists of Placement data of students in a XYZ campus. It includes secondary and higher secondary school percentage and specialization. It also includes degree specialization, type and Work experience and salary offers to the placed students}
\end{table}



\begin{table}
	\caption{ \texttt{9topcolleges2022.csv}
The list contains the top American colleges of 2022 and details about them including financial aid, student population, college phone number, website etc.
Most of the columns in the table are self explanatory.}
\end{table}


\begin{table}
	\caption{ \texttt{10boston.csv}
The Boston house price 
data of Harrison, D. and Rubinfeld, D.L. 
'Hedonic prices and the demand for clean air', 
J. Environ. Economics \& Management,
 vol.5, 81-102, 1978. Content }
%\begin{center}
	\begin{tabular}{l|l}\hline
CRIM & per capita crime rate by town \\
ZN & proportion of residential land zoned for lots over 25,000 sq.ft. \\
INDUS & proportion of non-retail business acres per town \\
CHAS & Charles River dummy variable (1 if tract bounds river; 0 otherwise) \\
NOX & nitric oxides concentration (parts per 10 million) [parts/10M] \\
RM & average number of rooms per dwelling \\
AGE & proportion of owner-occupied units built prior to 1940 \\
DIS & weighted distances to five Boston employment centres \\
RAD & index of accessibility to radial highways \\
TAX & full-value property-tax rate per \$10,000 [\$/10k] \\
PTRATIO & pupil-teacher ratio by town \\
B & The result of the equation $B=1000(Bk - 0.63)^2$, \\
& where Bk is the proportion of blacks by town \\
LSTAT & \\ % lower status of the population \\
MEDV & Median value of owner-occupied homes in $1000's [k]$ \\
	\hline
	\end{tabular}
%\end{center}
\end{table}

\newpage

\begin{thebibliography}{99}

\bibitem{Guns} Ayres, I., and Donohue, J.J. (2003). Shooting Down the ‘More Guns Less Crime’ Hypothesis. 
Stanford Law Review, 55, 1193–1312
	
\bibitem{Cigar} Baltagi B, Levin D (1992). Cigarette taxation: Raising revenues and reducing consumption. 
Structural Change and Economic Dynamics, 3(2), 321-335
	
\bibitem{Gasoline}
Baltagi, B.H. and Y.J. Griggin (1983) “Gasoline demand in the OECD: an application of pooling and 
testing procedures”, European Economic Review, 22.
	
\bibitem{Diamonds}
Chu, Singfat (2001) “Pricing the C’s of Diamond Stones”, Journal of Statistics Education, 9(2).
	
\bibitem{sleep75}
J.E. Biddle and D.S. Hamermesh (1990), “Sleep and the Allocation of Time,” 
Journal of Political Economy 98, 922-943.
	
\bibitem{wage2}
M. Blackburn and D. Neumark (1992), “Unobserved Ability, Efficiency Wages, and Interindustry Wage Differentials”, 
Quarterly Journal of Economics 107, 1421-1436.
	
\bibitem{SwissLabor}
Gerfin, M. (1996). Parametric and Semi-Parametric Estimation of the Binary Response Model of Labour Market Participation. 
Journal of Applied Econometrics, 11, 321–339.
	
\bibitem{Icecream}
Hildreth, C. and J. Lu (1960) Demand relations with autocorrelated disturbances, Technical Bul- letin No 2765, Michigan State University.
	
\bibitem{loanapp}
W.C. Hunter and M.B. Walker (1996), “The Cultural Affinity Hypothesis and Mortgage Lending Decisions”, 
Journal of Real Estate Finance and Economics 13, 57-70

\bibitem{Mroz}
Mroz, T. (1987) “The sensitivity of an empirical model of married women's hours of work to economic and statistical assumptions”, 
Econometrica, 55, 765-799.

\bibitem{LaborSupply}
Ziliak, Jim (1997) “Efficient Estimation With Panel Data when Instruments are Predetermined: An Empirical Comparison of Moment\-Condition Estimators”, 
Journal of Business and Economic Statistics, 419-431
	
\end{thebibliography}

\end{document}